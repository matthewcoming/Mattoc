\documentclass[conference]{IEEEtran}
\IEEEoverridecommandlockouts
% The preceding line is only needed to identify funding in the first footnote. If that is unneeded, please comment it out.
\usepackage{cite}
\usepackage{amsmath,amssymb,amsfonts}
\usepackage{algorithmic}
\usepackage{graphicx}
\usepackage{textcomp}
\usepackage{xcolor}
\def\BibTeX{{\rm B\kern-.05em{\sc i\kern-.025em b}\kern-.08em
    T\kern-.1667em\lower.7ex\hbox{E}\kern-.125emX}}
\begin{document}

\title{A Sample Paper\\
}

\author{\IEEEauthorblockN{1\textsuperscript{st} Matthew Coming}
\IEEEauthorblockA{\textit{Texas A\&M University } \\
\textit{Engineering and Computer Science Department}\\
College Station, Texas, USA} \\
\and
\IEEEauthorblockN{2\textsuperscript{nd} Brian W. Kernighan}
\IEEEauthorblockA{\textit{Princeton University} \\
\textit{Electrical Engineering Department}\\
Princeton, New Jersey, USA} \\
\and
\IEEEauthorblockN{3\textsuperscript{rd} }
\IEEEauthorblockA{\textit{Harvard University}\\
\textit{Applied Physics Department}}\\
}
\maketitle

\begin{abstract}
This document was created to learn how to create a \LaTeX\ article using the IEEE article template IEEEtran.
\end{abstract}

\begin{IEEEkeywords}
This, Is, How, I, Am, Listing-keywords
\end{IEEEkeywords}

\section{Introduction}\label{Intro}
This document is a model and instructions for \LaTeX\footnote{This is the first footnote}.
This section is numbered automatically with the roman numeral \ref{Intro} because it is my first use of the section command.

Furthermore, I did not type the roman numeral. Instead, I used \textbackslash label\{Intro\}\footnotemark to create an 'alias' to the Introduction section number, and then referenced the number by using \textbackslash ref\{Intro\}\footnotemark[\value{footnote}].

\footnotetext{in order to write these two phrases, I used \texttt{\textbackslash textbackslash \textit{command} \textbackslash \{\textit{marker}\textbackslash \} }}

The section command takes one argument, and IEEEtran formats the argument, \textbf{\textit{'Introduction'}}, as the title of the sectioin.

\section{Ease of Use}\label{Ease}

\subsection{My First Subsection}\label{first}
This is a subsection named \textit{'My First Subsection'} that sits within the section \textbf{\textit{Ease of Use}}. It should be sectioned by \textit{\ref{first}} and the roman numeral \ref{Ease}

\subsection{Meaning of Life}\label{meaning}
This subsection is named \textit{'Meaning of Life'} due to its enumeration being \ref{meaning}. Some would say that the meaning of life is to realize one's true purpose or potential, such as mastering \LaTeX or getting a freshman internship.

\section{Math}
Math in Latex takes two main forms. 

\subsection{equations}
The first, and most useful for this style of paper, is a \textbackslash begin\{equation\} command. Seen below, it spaces and fonts an equation to emphasize its importance. It is also numbered, and can be referenced just line sections and subsections with the combination of the \textbackslash label\{\textit{marker}\} and \textbackslash ref\{\textit{marker}\} commands.

\begin{equation}
a+b=\gamma\label{eq}
\end{equation}

\begin{equation}
c-d=\aleph\label{eq2}
\end{equation}

\subsection{In-Line}
The second flavor of Math in Latex is in-line. By using the math mode, you can show that $a+b=\gamma$ is (\ref{eq}) and $c-d=\aleph$ is (\ref{eq2}) without breaking the text. This is done simply by typing \$left side=right side\$.


\section{\LaTeX\ as a tool}% The backslash after LaTeX is to force one inter word space that the \LaTeX command

\subsection{Items and Love}
\begin{itemize}
        \item Here is an example of a bulleted list.
        \item One might use this to list all of the ways they love \LaTeX
                \begin{enumerate}
                        \item It automagically indents nested lists, even unumerations inside unordered.
                        \item It keeps track of how many equations you've made, and labels it for you.
                        \item By using Latex (and the hard work of some nerd at IEEE), you can make even the worst crap ever written look absolutely gorgeous.
                \end{enumerate}
\end{itemize}

\subsection{Some Common Mistakes}\label{SCM}
The good\footnotemark people at IEEE are extremely opinionated over what \textit{exactly} is considered good writing. Here are their suggestions.

\footnotetext{In Plato, perfect goodness is the Form of the Good, upon which everything that has being is ontologically and causally dependent. In Aristotle, the good is identified with the end or purpose of a natural being. The good is that towards which all things move for the fulfilment of their natures. By the time of Aquinas, medieval philosophers had identified the good in both the Platonic and Aristotelian senses with the Christian God and had argued that God is both the perfectly good creative source and the perfectly good end of all beings other than himself.}

\begin{itemize}
\item The word ``data'' is plural, not singular.
\item The subscript for the permeability of vacuum $\mu_{0}$, and other common scientific constants, is zero with subscript formatting, not a lowercase letter ``o''.
\item In American English, commas, semicolons, periods, question and exclamation marks are located within quotation marks only when a complete thought or name is cited, such as a title or full quotation. When quotation marks are used, instead of a bold or italic typeface, to highlight a word or phrase, punctuation should appear outside of the quotation marks. A parenthetical phrase or statement at the end of a sentence is punctuated outside of the closing parenthesis (like this). (A parenthetical sentence is punctuated within the parentheses.)
\item A graph within a graph is an ``inset'', not an ``insert''. The word alternatively is preferred to the word ``alternately'' (unless you really mean something that alternates).
\item Do not use the word ``essentially'' to mean ``approximately'' or ``effectively''.
\item In your paper title, if the words ``that uses'' can accurately replace the word ``using'', capitalize the ``u''; if not, keep using lower-cased.
\item Be aware of the different meanings of the homophones ``affect'' and ``effect'', ``complement'' and ``compliment'', ``discreet'' and ``discrete'', ``principal'' and ``principle''.
\item Do not confuse ``imply'' and ``infer''.
\item The prefix ``non'' is not a word; it should be joined to the word it modifies, usually without a hyphen.
\item There is no period after the ``et'' in the Latin abbreviation ``et al.''.
\item The abbreviation ``i.e.'' means ``that is'', and the abbreviation ``e.g.'' means ``for example''.
\end{itemize}

\subsection{Figures and Tables}
\paragraph{Positioning Figures and Tables} Place figures and tables at the top and 
bottom of columns. Avoid placing them in the middle of columns. Large 
figures and tables may span across both columns. Figure captions should be 
below the figures; table heads should appear above the tables. Insert 
figures and tables after they are cited in the text. Use the abbreviation 
``Fig.~\ref{fig}'', even at the beginning of a sentence.

\begin{table}[htbp]
\caption{Table Type Styles}
\begin{center}
\begin{tabular}{|c|c|c|c|}
\hline
\textbf{Table}&\multicolumn{3}{|c|}{\textbf{Table Column Head}} \\
\cline{2-4} 
\textbf{Head} & \textbf{\textit{Table column subhead}}& \textbf{\textit{Subhead}}& \textbf{\textit{Subhead}} \\
\hline
copy& More table copy$^{\mathrm{a}}$& &  \\
\hline
\multicolumn{4}{l}{$^{\mathrm{a}}$Sample of a Table footnote.}
\end{tabular}
\label{tab1}
\end{center}
\end{table}

\begin{figure}[htbp]
\centerline{\includegraphics{fig1.png}}
        \caption{Example\footnotemark of a figure caption.}
\label{fig}
\end{figure}

\footnotetext{Fear not, Charles Joseph Minard, I only intend to show the possible \textit{existence} of a figure, not an actual one}

\section*{References}

Please number citations consecutively within brackets \cite{b1}. The 
sentence punctuation follows the bracket \cite{b2}. Refer simply to the reference 
number, as in \cite{b3}---do not use ``Ref. \cite{b3}'' or ``reference \cite{b3}'' except at 
the beginning of a sentence: ``Reference \cite{b3} was the first $\ldots$''

Number footnotes separately in superscripts. Place the actual footnote at 
the bottom of the column in which it was cited. Do not put footnotes in the 
abstract or reference list. Use letters for table footnotes.

Unless there are six authors or more give all authors' names; do not use 
``et al.''. Papers that have not been published, even if they have been 
submitted for publication, should be cited as ``unpublished'' \cite{b4}. Papers 
that have been accepted for publication should be cited as ``in press'' \cite{b5}. 
Capitalize only the first word in a paper title, except for proper nouns and 
element symbols.

For papers published in translation journals, please give the English 
citation first, followed by the original foreign-language citation \cite{b6}.

\begin{thebibliography}{00}
\bibitem{b1} G. Eason, B. Noble, and I. N. Sneddon, ``On certain integrals of Lipschitz-Hankel type involving products of Bessel functions,'' Phil. Trans. Roy. Soc. London, vol. A247, pp. 529--551, April 1955.
\bibitem{b2} J. Clerk Maxwell, A Treatise on Electricity and Magnetism, 3rd ed., vol. 2. Oxford: Clarendon, 1892, pp.68--73.
\bibitem{b3} I. S. Jacobs and C. P. Bean, ``Fine particles, thin films and exchange anisotropy,'' in Magnetism, vol. III, G. T. Rado and H. Suhl, Eds. New York: Academic, 1963, pp. 271--350.
\bibitem{b4} K. Elissa, ``Title of paper if known,'' unpublished.
\bibitem{b5} R. Nicole, ``Title of paper with only first word capitalized,'' J. Name Stand. Abbrev., in press.
\bibitem{b6} Y. Yorozu, M. Hirano, K. Oka, and Y. Tagawa, ``Electron spectroscopy studies on magneto-optical media and plastic substrate interface,'' IEEE Transl. J. Magn. Japan, vol. 2, pp. 740--741, August 1987 [Digests 9th Annual Conf. Magnetics Japan, p. 301, 1982].
\bibitem{b7} M. Young, The Technical Writer's Handbook. Mill Valley, CA: University Science, 1989.
\end{thebibliography}
\vspace{12pt}
\color{red}
IEEE conference templates contain guidance text for composing and formatting conference papers. Please ensure that all template text is removed from your conference paper prior to submission to the conference. Failure to remove the template text from your paper may result in your paper not being published.

\end{document}
